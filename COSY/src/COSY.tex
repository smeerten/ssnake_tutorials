% Copyright 2016 - 2018 Bas van Meerten and Wouter Franssen
%
%This file is part of ssNake.
%
%ssNake is free software: you can redistribute it and/or modify
%it under the terms of the GNU General Public License as published by
%the Free Software Foundation, either version 3 of the License, or
%(at your option) any later version.
%
%ssNake is distributed in the hope that it will be useful,
%but WITHOUT ANY WARRANTY; without even the implied warranty of
%MERCHANTABILITY or FITNESS FOR A PARTICULAR PURPOSE.  See the
%GNU General Public License for more details.
%
%You should have received a copy of the GNU General Public License
%along with ssNake. If not, see <http://www.gnu.org/licenses/>.

\documentclass[11pt,a4paper]{article}
\include{DeStijl}

\usepackage[bitstream-charter]{mathdesign}
\usepackage[T1]{fontenc}
\usepackage[protrusion=true,expansion,tracking=true]{microtype}
\usepackage{siunitx}

%Set section font
\usepackage{sectsty}
\allsectionsfont{\color{black!70}\fontfamily{SourceSansPro-LF}\selectfont}
%--------------------


%Set toc fonts
\usepackage{tocloft}
%\renewcommand\cftchapfont{\fontfamily{SourceSansPro-LF}\bfseries}
\renewcommand\cfttoctitlefont{\color{black!70}\Huge\fontfamily{SourceSansPro-LF}\bfseries}
\renewcommand\cftsecfont{\fontfamily{SourceSansPro-LF}\selectfont}
%\renewcommand\cftchappagefont{\fontfamily{SourceSansPro-LF}\bfseries}
\renewcommand\cftsecpagefont{\fontfamily{SourceSansPro-LF}\selectfont}
\renewcommand\cftsubsecfont{\fontfamily{SourceSansPro-LF}\selectfont}
\renewcommand\cftsubsecpagefont{\fontfamily{SourceSansPro-LF}\selectfont}
%--------------------

%Define header/foot
%\usepackage{fancyhdr}
%\pagestyle{fancy}
%\fancyhead[LE,RO]{\fontfamily{SourceSansPro-LF}\selectfont \thepage}
%\fancyhead[LO,RE]{\fontfamily{SourceSansPro-LF}\selectfont \leftmark}
%\fancyfoot[C]{}
%--------------------

%remove page number from first chapter page
%\makeatletter
%\let\ps@plain\ps@empty
%\makeatother
%----------------------

\usepackage[hidelinks,colorlinks,allcolors=black, pdftitle={COSY processing in ssNake},pdfauthor={Wouter M.J.\ Franssen}]{hyperref}

\interfootnotelinepenalty=10000 %prevents splitting of footnote over multiple pages
\linespread{1.2}

\title{\color{black}\fontfamily{SourceSansPro-LF}\bfseries COSY processing in ssNake}
\author{}
\date{\color{black}\fontfamily{SourceSansPro-LF}\bfseries \today}


\begin{document}
%\newgeometry{left=72pt,right=72pt,top=95pt,bottom=95pt,footnotesep=0.5cm}
\microtypesetup{protrusion=true} % enables protrusion

\maketitle

\section{Introduction}
A correlation spectroscopy (COSY) experiment is one of the most common 2D NMR experiments. It is
used to visualise through-bond connections in molecules (via J-couplings). In the 2D spectrum,
cross-peaks appear between lines which are J-coupled. As J-couplings are hard to observe in the
solid state, COSY experiments are usually only recorded for liquid state samples. In this tutorial,
we will show how to process magnitude COSY data (a phase sensitive COSY should be processed
differently).



\section{Data}
The data for this tutorial is a magnitude type $^1$H--$^1$H COSY spectrum of a mixture of toluene
and ethyl pyruvate (1:1) recorded on a \SI{1}{\tesla} (\SI{43}{\MHz}) Magritek device. In this case,
we expect to see cross-peaks between the two lines of toluene, and between the CH$_3$ and CH$_2$
resonances of ethyl pyruvate (the base methyl group of ethyl pyruvate overlaps with the methyl
group of toluene). The 1D $^1$H spectrum of this sample is shown below.
\begin{center}
\includegraphics[width=0.8\linewidth]{Figs/1D.pdf}
\end{center}

\section{Processing the data}
In principle, processing a magnitude type COSY spectrum is no big deal. All that is required is a
Fourier transform along both axes.


\begin{itemize}
  \item Open the Magritek file using \texttt{File $\longrightarrow$ Open}
  and navigate to the `cosy' folder. Opening one of the files here opens the data
  \item Fourier transform via the `Fourier' button
\end{itemize}
This should give:
\begin{center}
\includegraphics[width=0.8\linewidth]{Figs/Fig1.png}
\end{center}
Note that the line we see here have a very bad phase. This is not important, as the final view of
this data only means something in absolute mode (we will do this later on). We also need a Fourier
transform in the indirect dimension:

\begin{itemize}
  \item Switch to viewing D1 by clicking on its radio button in the sideframe
\item Do a complex conjugate via \texttt{Tools $\longrightarrow$ Complex Conjugate} \footnote{Doing
	 a complex conjugate flips the resulting spectrum. This is needed in ssNake, as Magritek uses a
	 different definition for the rotation direction of the magnetization. For the direct dimension,
	 this is always necessary, and is done by ssNake upon loading the data. If the indirect dimension
  is also a spectral axis, you need to do this manually for this axis. If you do not do this, the
resulting spectrum will be mirrored.}
  \item Fourier transform via the `Fourier' button
  \item Switch back to viewing D2 by clicking on its radio button in the sideframe
\end{itemize}

Now that we have processed both dimension, we want to view the spectrum as a contour plot:
\begin{itemize}
  \item Switch to view to a contour plot via \texttt{Plot $\longrightarrow$ Contour}
\end{itemize}
This should show (zoomed in):
\begin{center}
\includegraphics[width=0.8\linewidth]{Figs/Fig2.png}
\end{center}
We are not seeing much, but the lower contour cutoff is at 10\%, so we should lower this to see
more:
\begin{itemize}
  \item Set the minimum contour limit to 1\% (in the sideframe, under `Contour
	 limits')\footnote{Note that you can also `zoom' the contours by scrolling the mouse button while
	 holing the `shift' button}
\end{itemize}
This shows:
\begin{center}
\includegraphics[width=0.8\linewidth]{Figs/Fig3.png}
\end{center}
As you can see, we have some issues with the phase. This is because we should view this data in
absolute mode:
\begin{itemize}
  \item Set the plot to absolute (bottom frame under `Plot' set to `Abs')
\end{itemize}
This gives:
\begin{center}
\includegraphics[width=0.8\linewidth]{Figs/Fig4.png}
\end{center}
which starts to look like a COSY! To better see what are cross peaks, and which lines are on the
diagonal, we can ask ssNake to plot a line on the diagonal:
\begin{itemize}
  \item Turn on the diagonal by ticking the box in the sideframe
\end{itemize}
This shows:
\begin{center}
\includegraphics[width=0.8\linewidth]{Figs/Fig5.png}
\end{center}
Which is our final spectrum. You can see some `star' patterns in this spectrum, especially for the
strong toluene lines. This come from the fact to we view the data in absolute mode (which is
unavoidable for this experiment). The next section will describe how we can reduce these. 


\section{Sine square apodization}
In order to reduce the `star' artifacts in a COSY experiment, we can use some special apodization
along both D2 and D1. For this, we do the same processing as before, but before each Fourier
transform we do a sine square apodization (which we do in ssNake with a 90$\circ$ phase shifted
cosine square apodization):
\begin{itemize}
  \item Apodize with Cos\^{}2 with frequency 2, and phase 90 (\texttt{Tools $\longrightarrow$ Apodize})
\end{itemize}
If we do this along D2, we should see something like this:
\begin{center}
\includegraphics[width=0.8\linewidth]{Figs/Fig6.png}
\end{center}
Doing this along both D2 and D1 leads (after performing all the steps described above) to the
following 2D spectrum:
\begin{center}
\includegraphics[width=0.8\linewidth]{Figs/Fig7.png}
\end{center}
Which shows a strong reduction of the `star' effect.


\end{document}
