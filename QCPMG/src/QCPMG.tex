% Copyright 2016 - 2017 Bas van Meerten and Wouter Franssen
%
%This file is part of ssNake.
%
%ssNake is free software: you can redistribute it and/or modify
%it under the terms of the GNU General Public License as published by
%the Free Software Foundation, either version 3 of the License, or
%(at your option) any later version.
%
%ssNake is distributed in the hope that it will be useful,
%but WITHOUT ANY WARRANTY; without even the implied warranty of
%MERCHANTABILITY or FITNESS FOR A PARTICULAR PURPOSE.  See the
%GNU General Public License for more details.
%
%You should have received a copy of the GNU General Public License
%along with ssNake. If not, see <http://www.gnu.org/licenses/>.

\documentclass[11pt,a4paper]{article}
\include{DeStijl}

\usepackage[bitstream-charter]{mathdesign}
\usepackage[T1]{fontenc}
\usepackage[protrusion=true,expansion,tracking=true]{microtype}
\pgfplotsset{compat=1.7,/pgf/number format/1000 sep={}, axis lines*=left,axis line style={gray},every outer x axis line/.append style={-stealth'},every outer y axis line/.append style={-stealth'},tick label style={font=\small},label style={font=\small},legend style={font=\footnotesize}}
\usepackage{colortbl}
\usetikzlibrary{calc}

%Set section font
\usepackage{sectsty}
\allsectionsfont{\color{black!70}\fontfamily{SourceSansPro-LF}\selectfont}
%--------------------


%Set toc fonts
\usepackage{tocloft}
%\renewcommand\cftchapfont{\fontfamily{SourceSansPro-LF}\bfseries}
\renewcommand\cfttoctitlefont{\color{black!70}\Huge\fontfamily{SourceSansPro-LF}\bfseries}
\renewcommand\cftsecfont{\fontfamily{SourceSansPro-LF}\selectfont}
%\renewcommand\cftchappagefont{\fontfamily{SourceSansPro-LF}\bfseries}
\renewcommand\cftsecpagefont{\fontfamily{SourceSansPro-LF}\selectfont}
\renewcommand\cftsubsecfont{\fontfamily{SourceSansPro-LF}\selectfont}
\renewcommand\cftsubsecpagefont{\fontfamily{SourceSansPro-LF}\selectfont}
%--------------------

%Define header/foot
%\usepackage{fancyhdr}
%\pagestyle{fancy}
%\fancyhead[LE,RO]{\fontfamily{SourceSansPro-LF}\selectfont \thepage}
%\fancyhead[LO,RE]{\fontfamily{SourceSansPro-LF}\selectfont \leftmark}
%\fancyfoot[C]{}
%--------------------

%remove page number from first chapter page
%\makeatletter
%\let\ps@plain\ps@empty
%\makeatother
%----------------------

\usepackage[hidelinks,colorlinks,allcolors=black, pdftitle={QCPMG},pdfauthor={Wouter M.J.\ Franssen}]{hyperref}

\interfootnotelinepenalty=10000 %prevents splitting of footnote over multiple pages
\linespread{1.2}

\title{\color{black}\fontfamily{SourceSansPro-LF}\bfseries QCPMG processing in ssNake}
\author{}
\date{\color{black}\fontfamily{SourceSansPro-LF}\bfseries \today}


\begin{document}
%\newgeometry{left=72pt,right=72pt,top=95pt,bottom=95pt,footnotesep=0.5cm}
\microtypesetup{protrusion=true} % enables protrusion

\maketitle

\section{Introduction}
This tutorial will explain how QCPMG NMR data can be processed with ssNake.
The tutorial provided with the ssNake program is considered as prior knowledge.
If you have not yet studied this, please do so before continuing with this example.

QCPMG stands for Quadrupolar Carr-Purcell-Meiboom-Gill.
The sequence is used to record a series of consecutive echoes.
Via careful processing the signal-to-noise ratio of this data can be higher than when only a single echo would have been recorded.
The experiment is particularly helpful for broad quadrupolar lineshapes, where the decay of the signal is much faster than the loss of magnetization by relaxation ($T_2$).
In this case, a large number of echoes can be recorded in one go.

QCPMG experiments can be processed in a number of ways to produce either a spikelet pattern, or a regular NMR spectrum.
Both have advantages in specific circumstances.


\section{Data}
The data provided with this tutorial is a $^{35}$Cl spectrum of magnesium chloride (MgCl$_2$) recorded at 20~T using 15.6~kHz MAS.


\section{Processing}
\subsection{Sum echo spectrum}
The most general way of analysing the series of echoes that are recorded in a QCPMG experiment is by adding all the individual echoes, and process it as you would a single echo experiment.
This results in a spectrum which is easy to interpret and has a higher signal-to-noise ratio compared to a regular echo experiment.
However, processing can be a bit tricky.

The data supplied in this tutorial has 137 recorded echoes, consisting of 1088 data points each.
In order to sum these, we have to split the data in 137 parts to create a pseudo 2D data set with a shape of $137 \times 1088$.
The number of echoes is chosen when setting up the experiment, but can also be determined afterwards.
Determining the number of points per echo can be more tricky.
Dividing the total number of data points by 137 should provide this number, but in our case the QCPMG data has zeroes appended to the echoes.
What we do know is that only with the correct number of points (1088) the splitting of the data results in aligned echos.

\begin{itemize}
  \item Open the data in the \texttt{Raw} directory
\end{itemize}
Each echo has some zeroes appended to it, which is a consequence of how the Varian equipment acquires the QCPMG data.
This can be seen in the following Figure (from the start of the FID):
\begin{center}
\includegraphics[width=0.7\linewidth]{Figs/Fig1.png}
\end{center}
which shows the first 2 and a half echoes. We now must split the data in the 137 echoes.
\begin{itemize}
  \item Set the size to $137 \cdot 1088 = 149056$ data points (Matrix $\longrightarrow$ Sizing)
  \item Split the data: Matrix $\longrightarrow$ Split (137 sections)
\end{itemize}
Now we have all the separate echoes.
Viewing D2 shows:
\begin{center}
\includegraphics[width=0.7\linewidth]{Figs/Fig2.png}
\end{center}
which is the first echo.
Scrolling through D1, we can see that for each echo, the position is the same: the splitting has been performed correctly.
A stack plot of the first 10 echoes for example looks like this:
\begin{center}
\includegraphics[width=0.7\linewidth]{Figs/Fig3.png}
\end{center}
If we had used 1000 points per echo we would see:
\begin{center}
\includegraphics[width=0.7\linewidth]{Figs/Fig4.png}
\end{center}
which is clearly very, very, wrong.
A good check is to overlay the first and last echo. The `zero' part at the end should be at the same position.

\subsubsection{Option 1: Directly summing the echoes}
Using the properly split data we can sum the echoes:
\begin{itemize}
  \item Go to D1 (using the radio button in the side frame)
  \item Sum along this dimension using: Matrix $\longrightarrow$ Region $\longrightarrow$ Sum (with no input, just push Ok).
\end{itemize}
This results in:
\begin{center}
\includegraphics[width=0.7\linewidth]{Figs/Fig5.png}
\end{center}

We will now process the data using whole echo processing.
More on this subject can be found in the Whole Echo processing tutorial.
In short:
\begin{itemize}
  \item Swap the echo using Tools $\longrightarrow$ Swap echo at point 512
  \item Zero fill to 8192 points using Matrix $\longrightarrow$ Sizing
  \item Fourier transform with the Fourier button in the bottom frame
  \item Phase a bit using Tools $\longrightarrow$ Phasing (0th order: 0.220, 1st order: 68)
  \item 800 Hz Gaussian apodization (Tools $\longrightarrow$ Apodize)
\end{itemize}
This should result in (zoomed):
\begin{center}
\includegraphics[width=0.7\linewidth]{Figs/Fig6.png}
\end{center}
Which is the final spectrum using this processing method.
It is saved as \texttt{EchoSumOption2.mat} in the data directory.

\subsubsection{Option 2: T2 weighting}
Summing all the echoes as has been done in Option 1 is fine in some cases, but could lead to a suboptimal signal-to-noise ratio.
Imagine for instance that the signal would have almost completely decayed at the last echo. 
Clearly, including this echo in the sum will only increase the noise.
So we would like to add the echoes with a weighting factor which equals the amount of signal present in the echo.
As the echo intensity is reducing due to $T_2$ relaxation, this scaling method is referred to as `$T_2$ weighting'.

To perform $T_2$  weighting, we should first obtain the $T_2$ value.
We start with the data which we have just split in 137 parts (before the Option 1 section processing).
The echo maxima are in this case located at data point 514.
Put this number in the D2 box and press enter.
This should result in a curve like this:
\begin{center}
\includegraphics[width=0.7\linewidth]{Figs/Fig7.png}
\end{center}

We now want to fit this line with a $T_2$ decay curve.
Do note that the spectral width in this dimension is not sensible (as it was copied from the other dimension when we split the data).
This means that our fitted $T_2$ will be in arbitrary units, but as long as we apply the weighting in the same units, we should be fine\footnote{Alternatively, we could define the spectral width as the inverse of the time between the echo tops.}.
Let's fit a $T2$ curve:

\begin{itemize}
\item Open the relaxation fitting window (Fitting $\longrightarrow$ Relaxation Curve)
\item Set the `Constant' to 0, and the `Coefficient' to 1
\item Set the `T' variable to 0.001 as an initial guess
\item Tick the boxes next to the `Constant' and `Coefficient' (this fixes them, so that they are not optimized during the fitting)
\item Click `Fit' (and perhaps a second time to see if the result changes significantly)
\end{itemize}
This results in:
\begin{center}
\includegraphics[width=0.7\linewidth]{Figs/Fig8.png}
\end{center}
with a $T_2$ of 0.0004864 seconds.
Close the Window using the `Cancel' button

We now would like to use this $T_2$ curve to scale our echos.
We will do this using Lorentzian apodization, which has the same function as a $T_2$ decay (exponential decay).
The value that we will use is: $\text{LB} = 1/(T_2 \cdot \pi) = 654$ Hz.


\begin{itemize}
  \item Use Tools $\longrightarrow$ Apodize and apply 654 Hz Lorentzian apodization along D1
\end{itemize}
The data can now be further processed using the methods of Option 1 above.
This $T_2$ weighted spectrum has a better signal-to-noise ratio that that of Option 1, although in this particular example the data has only high-intensity echoes, so the difference will be small.
The final spectrum is saved in the data directory as \texttt{EchoSumOption2.mat}.



\subsection{Spikelet spectrum}
A wholly different method for processing the data from a QCPMG experiment is the spikelet method.
It features a direct Fourier transform of the FID, with no splitting and summing of the echoes.
The resulting spectrum has a series of spikes (i.e. spikelets) in the spectrum, with a distance of $1 /T$, with $T$ the time between two echoes.
The advantage of the spikelet method is that all the signal is concentrated in the spikes, leading to a huge increase in signal-to-noise ratio.
The disadvantage is that, while the maxima of the spikes follow the intensity distribution of the `regular' echo spectrum, the area between the spikes has no information.
If only a couple of spikelets are present, the shape of the quadrupolar line becomes obscured. 


\begin{itemize}
  \item Open the data in the \texttt{Raw} folder
	\item Zero fill to 524288 (Matrix $\longrightarrow$ Sizing)
\end{itemize}
We have to apply a first order phase correction, to make sure the FID starts at the first echo maximum.
The position of this top is data point 514.
To correct this we require $\theta = 360 \cdot n = 185040$ of first order phasing:


\begin{itemize}
  \item Phase with 185040 first order phasing (Tools $\longrightarrow$ Phasing)

\end{itemize}
And to obtain a decent spectrum:
\begin{itemize}
	\item Fourier transform (using the `Fourier' button)
	\item Gaussian apodization of 4 Hz
	\item Phase (0th order: -104, 1st order: -618)
\end{itemize}
The final spectrum should then look like: 
\begin{center}
\includegraphics[width=0.7\linewidth]{Figs/Fig9.png}
\end{center}
which is saved as \texttt{spikelets.mat} in the data folder.

When we performed the Gaussian apodization, we also removed the tail of the first echo (which was shifted to the end of the FID by the large first order phase correction).
This suppresses the formation of a regular echo line shape under our QCPMG spikelets at the cost of intensity.
In this case, we have many intense echoes, and the added baseline of this tail does not help us.
If we do wish to retain this, we must make sure that the apodization is performed \textit{before} the large first order phase shift.


\end{document}
