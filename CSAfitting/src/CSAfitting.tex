% Copyright 2016 - 2018 Bas van Meerten and Wouter Franssen
%
%This file is part of ssNake.
%
%ssNake is free software: you can redistribute it and/or modify
%it under the terms of the GNU General Public License as published by
%the Free Software Foundation, either version 3 of the License, or
%(at your option) any later version.
%
%ssNake is distributed in the hope that it will be useful,
%but WITHOUT ANY WARRANTY; without even the implied warranty of
%MERCHANTABILITY or FITNESS FOR A PARTICULAR PURPOSE.  See the
%GNU General Public License for more details.
%
%You should have received a copy of the GNU General Public License
%along with ssNake. If not, see <http://www.gnu.org/licenses/>.

\documentclass[11pt,a4paper]{article}
\include{DeStijl}

\usepackage[bitstream-charter]{mathdesign}
\usepackage[T1]{fontenc}
\usepackage[protrusion=true,expansion,tracking=true]{microtype}
\pgfplotsset{compat=1.7,/pgf/number format/1000 sep={}, axis lines*=left,axis line style={gray},every outer x axis line/.append style={-stealth'},every outer y axis line/.append style={-stealth'},tick label style={font=\small},label style={font=\small},legend style={font=\footnotesize}}
\usepackage{colortbl}
\usetikzlibrary{calc}

%Set section font
\usepackage{sectsty}
\allsectionsfont{\color{black!70}\fontfamily{SourceSansPro-LF}\selectfont}
%--------------------


%Set toc fonts
\usepackage{tocloft}
%\renewcommand\cftchapfont{\fontfamily{SourceSansPro-LF}\bfseries}
\renewcommand\cfttoctitlefont{\color{black!70}\Huge\fontfamily{SourceSansPro-LF}\bfseries}
\renewcommand\cftsecfont{\fontfamily{SourceSansPro-LF}\selectfont}
%\renewcommand\cftchappagefont{\fontfamily{SourceSansPro-LF}\bfseries}
\renewcommand\cftsecpagefont{\fontfamily{SourceSansPro-LF}\selectfont}
\renewcommand\cftsubsecfont{\fontfamily{SourceSansPro-LF}\selectfont}
\renewcommand\cftsubsecpagefont{\fontfamily{SourceSansPro-LF}\selectfont}
%--------------------

%Define header/foot
%\usepackage{fancyhdr}
%\pagestyle{fancy}
%\fancyhead[LE,RO]{\fontfamily{SourceSansPro-LF}\selectfont \thepage}
%\fancyhead[LO,RE]{\fontfamily{SourceSansPro-LF}\selectfont \leftmark}
%\fancyfoot[C]{}
%--------------------

%remove page number from first chapter page
%\makeatletter
%\let\ps@plain\ps@empty
%\makeatother
%----------------------

\usepackage[hidelinks,colorlinks,allcolors=black, pdftitle={CSA pattern fitting in ssNake},pdfauthor={Wouter M.J.\ Franssen}]{hyperref}

\interfootnotelinepenalty=10000 %prevents splitting of footnote over multiple pages
\linespread{1.2}

\title{\color{black}\fontfamily{SourceSansPro-LF}\bfseries CSA pattern fitting in ssNake}
\author{}
\date{\color{black}\fontfamily{SourceSansPro-LF}\bfseries \today}


\begin{document}
%\newgeometry{left=72pt,right=72pt,top=95pt,bottom=95pt,footnotesep=0.5cm}
\microtypesetup{protrusion=true} % enables protrusion

\maketitle

\section{Introduction}
This tutorial will explain how CSA NMR data can be fitted ssNake.
The tutorial delivered with the ssNake program is considered as prior knowledge. If you have not yet
studied this, please do so before continuing with this example.

In solid state NMR, the anisotropic chemical shift interaction (CSA) is often a relevant factor in the description of a spectrum.
While, using fast magic angle spinning, the isotropic shift can be easily established, the size and asymmetry of the anisotropy are usually somewhat harder to determine.
As these spectra can be simulated in a straightforward way, fitting the experimental data is a good way to determine these values.


\section{Data}
The data used in this tutorial is of glycine powder which was recorded on a Varian 300 MHz spectrometer.
The spectrum shows the $^{13}$C resonance of the carboxyl group of glycine.


\section{Fitting a static CSA pattern}
Begin by loading the processed data from the file named \texttt{GlycineStatic.mat} which is delivered with this tutorial.
As this is a chemical shift pattern, we want the x-axis to be in ppm:
\begin{itemize}
  \item In the bottomframe, set \texttt{Axis} to \texttt{ppm}
\end{itemize}
The spectrum should look like:
\begin{center}
\includegraphics[width=0.8\linewidth]{Figs/Fig1.png}
\end{center}
Which is a nice CSA powder pattern of a static sample. We now need to fit this data to extract the
CSA tensor parameters.
Go to \texttt{Fitting $\longrightarrow$ CSA}.

This displays the following window:
\begin{center}
\includegraphics[width=0.8\linewidth]{Figs/Fig2.png}
\end{center}

This window has quite some parameters.
The most important parameters are the boxes on the right: $\delta_{11}$, $\delta_{22}$, and $\delta_{33}$ are the CSA tensor values.
Also shown are some line broadening parameters (Lorentz and Gauss).
The tick boxes next to the input boxes indicate whether or not the parameters are fixed (checked) or that they are going to be optimized during fitting (unchecked).

Before we can perform an iterative fit, we must first supply some starting values for the fit.
This makes it easier for ssNake to find the best solution.
In this CSA fit, this can be easily done by left clicking on the spectrum: each click sets one of the $\delta$-values to the ppm value under the cursor.
In this case, these values should be equal to the `singularities' of the powder pattern.
Left click on the leftmost edge of the CSA pattern, the `peak' and the rightmost edge.
This should set the $\delta$-values to approximately 243, 181 and 106 ppm.

Clicking `Sim' in the left part of the window simulates a spectrum using our current settings.
The spectrum will then look like:
\begin{center}
\includegraphics[width=0.8\linewidth]{Figs/Fig3.png}
\end{center}
which looks very ugly.
Here the orange line represents the simulated spectrum.
The reason why it looks like this is because the line broadening is quite low (1 Hz), and we only simulate a finite number of powder orientations.
This leads to a series of peaks for each crystal orientation in our synthetic powder, and not a nice CSA pattern.
This issue can be overcome by increasing the number of powder orientations.
\begin{itemize}
  \item Change the Cheng\footnote{The Cheng number represents the number of orientation to be
	 simulated. It is an exponential scale, so the number of orientations increases rapidly with the Cheng number.} number from 15 to 25
\end{itemize}
Simulating again will show a spectrum like:
\begin{center}
\includegraphics[width=0.8\linewidth]{Figs/Fig4.png}
\end{center}
which looks a lot more like a powder pattern.
However, with Cheng at 25, the simulation starts to become slower due to the larger number of orientations.
In this case we actually do not need so many orientations: our actual spectrum is not so well-defined as there is significant line broadening present (you can see this at the rounded edges of the pattern).
Increasing the line broadening in the simulation also means we require less powder orientations:
\begin{itemize}
  \item Set Cheng back to 15
  \item Set Lorentz at 500 Hz
\end{itemize}
Clicking `Sim' now results in:
\begin{center}
\includegraphics[width=0.8\linewidth]{Figs/Fig5.png}
\end{center}
which matches the experimental spectrum quite well.

We are now ready to let ssNake do some fitting.
Click `Fit' to let it work its magic.
This should result in:
\begin{center}
\includegraphics[width=0.8\linewidth]{Figs/Fig6.png}
\end{center}
with tensor values 244.8, 180.2, and 106.1 ppm.
Often, the CSA tensor values are expressed in some different definition.
Common is the iso-aniso-eta definition.
We can tell ssNake we want to use a certain definition by switching from $\delta_{11}$--$\delta_{22}$--$\delta_{33}$ to $\delta_\text{iso}$--$\delta_\text{aniso}$--$\eta$.
ssNake automatically converts the tensor values.
This results in 177.0, $-70.93$, and 0.9107.
Using these other definitions can also be convenient during fitting, for example in cases were you known $\delta_\text{iso}$ or $\eta$, and you want to fix these values during the fit.
Do note that in some definitions the mouse cannot be used to set values by clicking in the spectrum.

\section{Fitting a MAS CSA pattern}
Naturally, ssNake can also fit a CSA pattern when it has been measured under MAS conditions.
This is performed in a similar way, although there are some additional things to take into account.

Start by loading \texttt{Glycine1kHzMAS.mat} which is delivered with this tutorial.
Again, put the axis on ppm before continuing.
Go to the fitting routine via \texttt{Fitting $\longrightarrow$ CSA}.
This shows:
\begin{center}
\includegraphics[width=0.8\linewidth]{Figs/Fig7.png}
\end{center}
In principal, we could use the mouse again to set the initial values of the tensor values.
However, in this spinning spectrum, these values are unclear.
But still, we can give a rough estimate of the tensor values.
We can start with the same values as for the static fit: 243, 181 and 106 ppm.

Before we can perform a useful simulation, we must tell ssNake that we are using MAS:
\begin{itemize}
  \item Set \texttt{MAS} to \texttt{Finite MAS}
  \item Set \texttt{Spin.\ speed} on 1 kHz
\end{itemize}
Pushing \texttt{Sim} now results in:
\begin{center}
\includegraphics[width=0.8\linewidth]{Figs/Fig8.png}
\end{center}
By default, the integral is set to the integral of the experimental spectrum.
In this case, the simulated spectrum looks quite large, but this is mostly due to the small line broadening values.
Let's increase these values.
\begin{itemize}
  \item Set \texttt{Lorentz} to 40 Hz
\end{itemize}
This results in a more reasonable spectrum.

When fitting this spectrum, it is important to realise that it is actually quite a difficult fit.
The very narrow lines are hard to position correctly using the iterative process.
It is best if we help the fitting a bit by fixing some values.
In this MAS spectrum, the isotropic shift has readily been determined.
In this case, it is at 176.54 ppm. In principle, this value does not need to be fit.
So lets fix it at this value:
\begin{itemize}
  \item Change the definition to iso--aniso--eta
  \item Increase the number of significant digits to 5 (see \texttt{Preferences} on the left)
	\item Set $\delta_\text{iso}$ to 176.54 and tick on the left of this value (to fix it)
	 \item Push \texttt{Sim}
\end{itemize}
The spectrum should now look like:
\begin{center}
\includegraphics[width=0.8\linewidth]{Figs/Fig9.png}
\end{center}
What is clear from this Figure, is that the spinning speed is not set correctly.
Clearly, the experimental spectrum was measured with slightly more than 1 kHz MAS.
We can manually correct this by looking at one of the peaks further away from the centre, and adjusting the speed till these sidebands overlap.
\begin{itemize}
  \item Set \texttt{Spin.\ speed} to 1.0590 kHz
\end{itemize}
Now, we are ready to fit!
Pushing \texttt{Fit} optimizes all unticked parameters.
This leads to a reasonable fit.
Now that we are close to the optimum, we can also `release' the $\delta_\text{iso}$ and spin speed parameters by unticking their boxes.
Pushing Fit again optimizes all the values.
This should lead to values close to 176.58, $-69.457$, and 0.91502 for iso, aniso and eta.
Which is pretty close to the values we got for the static spectrum.
Note that we did not touch the \texttt{\# sidebands} parameter.
This should always be larger than the number of observed sidebands in the simulation. In this example the default value of 32 is sufficient.




\end{document}
