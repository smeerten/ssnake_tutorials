% Copyright 2016 - 2018 Bas van Meerten and Wouter Franssen
%
%This file is part of ssNake.
%
%ssNake is free software: you can redistribute it and/or modify
%it under the terms of the GNU General Public License as published by
%the Free Software Foundation, either version 3 of the License, or
%(at your option) any later version.
%
%ssNake is distributed in the hope that it will be useful,
%but WITHOUT ANY WARRANTY; without even the implied warranty of
%MERCHANTABILITY or FITNESS FOR A PARTICULAR PURPOSE.  See the
%GNU General Public License for more details.
%
%You should have received a copy of the GNU General Public License
%along with ssNake. If not, see <http://www.gnu.org/licenses/>.

\documentclass[11pt,a4paper]{article}
\include{DeStijl}

\usepackage[bitstream-charter]{mathdesign}
\usepackage[T1]{fontenc}
\usepackage[protrusion=true,expansion,tracking=true]{microtype}
\pgfplotsset{compat=1.7,/pgf/number format/1000 sep={}, axis lines*=left,axis line style={gray},every outer x axis line/.append style={-stealth'},every outer y axis line/.append style={-stealth'},tick label style={font=\small},label style={font=\small},legend style={font=\footnotesize}}
\usepackage{colortbl}
\usetikzlibrary{calc}

%Set section font
\usepackage{sectsty}
\allsectionsfont{\color{black!70}\fontfamily{SourceSansPro-LF}\selectfont}
%--------------------


%Set toc fonts
\usepackage{tocloft}
%\renewcommand\cftchapfont{\fontfamily{SourceSansPro-LF}\bfseries}
\renewcommand\cfttoctitlefont{\color{black!70}\Huge\fontfamily{SourceSansPro-LF}\bfseries}
\renewcommand\cftsecfont{\fontfamily{SourceSansPro-LF}\selectfont}
%\renewcommand\cftchappagefont{\fontfamily{SourceSansPro-LF}\bfseries}
\renewcommand\cftsecpagefont{\fontfamily{SourceSansPro-LF}\selectfont}
\renewcommand\cftsubsecfont{\fontfamily{SourceSansPro-LF}\selectfont}
\renewcommand\cftsubsecpagefont{\fontfamily{SourceSansPro-LF}\selectfont}
%--------------------

%Define header/foot
%\usepackage{fancyhdr}
%\pagestyle{fancy}
%\fancyhead[LE,RO]{\fontfamily{SourceSansPro-LF}\selectfont \thepage}
%\fancyhead[LO,RE]{\fontfamily{SourceSansPro-LF}\selectfont \leftmark}
%\fancyfoot[C]{}
%--------------------

%remove page number from first chapter page
%\makeatletter
%\let\ps@plain\ps@empty
%\makeatother
%----------------------

\usepackage[hidelinks,colorlinks,allcolors=black, pdftitle={REDOR processing in ssNake},pdfauthor={Wouter M.J.\ Franssen}]{hyperref}

\interfootnotelinepenalty=10000 %prevents splitting of footnote over multiple pages
\linespread{1.2}

\title{\color{black}\fontfamily{SourceSansPro-LF}\bfseries REDOR processing in ssNake}
\author{}
\date{\color{black}\fontfamily{SourceSansPro-LF}\bfseries \today}


\begin{document}
%\newgeometry{left=72pt,right=72pt,top=95pt,bottom=95pt,footnotesep=0.5cm}
\microtypesetup{protrusion=true} % enables protrusion

\maketitle

\section{Introduction}
This tutorial will explain how rotational-echo double-resonance (REDOR) NMR data can be processed in ssNake.
The tutorial provided with the ssNake program is considered as prior knowledge.
If you have not yet studied this, please do so before continuing with this example.

In a REDOR experiment, echo spectra are recorded with an increase in echo time $\tau$.
As the experiment is recorded while magic angle spinning, the echo time must be multiple of the rotor period $1/\nu_\text{r}$.
In the experiment, the spinning averages the dipolar interactions.
The goal of a REDOR experiment is to obtain information on the dipolar couplings between nuclei of two different types, while still having the enhanced resolution of MAS.
This is done by performing echo experiments twice: with and without 180 degree pulses on the other nuclei.
These pulses reintroduce (recouple) the dipolar coupling, causing a dephasing during the echo time.
Essentially, this is a measure of the effective $T_2^*$ polarization loss in the presence or absence of dipolar coupling.

Measuring REDOR data results in two data sets: one with, and one without the recoupling pulses.
Both these data sets are an array of 1D spectra, with increasing echo times (number of rotor periods).
For each echo time, we determine the integral (or height) of a peak.
We thus obtain two integrals of the relevant peak: without pulse ($S_\text{wo}$) and with pulses ($S_\text{wi}$).
The relevant information we require is the relative difference between these:
\begin{equation*}
  \Delta S = \frac{S_\text{wo} - S_\text{wi}}{S_\text{wo}}
\end{equation*}
A REDOR plot shows this $\Delta S$ as a function of echo time (or rotor periods).


\section{Data}
The data in this tutorial was recorded using a Varian 400 MHz spectrometer on glycine using 5 kHz MAS.
The data was recorded with 2 to 40 rotor periods per echo delay (in steps of 2).



\section{Processing}
Processing REDOR data is not difficult.
Only a couple of steps are required.

\begin{itemize}
\item Open the REDOR\_no\_pulse.fid data using File $\longrightarrow$ Open
\item Zerofill to 16384 points (Matrix  $\longrightarrow$ Sizing)
\item Fourier Transform
\item Phase with $-6.0$ degrees 0 order phasing (Tools $\longrightarrow$ Phasing)
\end{itemize}
This results in the following spectrum:
\begin{center}
\includegraphics[width=0.8\linewidth]{Figs/Fig1.png}
\end{center}
In this case, we are interested in the right peak (CH$_2$).
We want a plot of the integral of this line over all the 11 spectra.
For this, we integrate the peak.
Note that ssNake has two types of integral: via Fitting  $\longrightarrow$ Integrals, which can be seen as a way to ask ssNake `what is the integral of this peak'.
Or via Matrix $\longrightarrow$ Region $\longrightarrow$
Integrate, which integrates a region across all traces and sets the new data to the integrals.
In this case, we require the latter type of integration.

\begin{itemize}
\item Use Matrix $\longrightarrow$ Region $\longrightarrow$ Integrate and click on the left and right side of the CH$_2$ peak (ignore the tiny peak next to it for the moment). For example, positions 5679 and 5861.
\end{itemize}
This should result in a curve like:
\begin{center}
\includegraphics[width=0.8\linewidth]{Figs/Fig2.png}
\end{center}
This shows how the integral decays in the experiment without the pulses (dipolar coupling disabled).

Now, we do the same for the data with the pulses. 
\begin{itemize}
\item Open the REDOR\_with\_pulse.fid data using File $\longrightarrow$ Open
\end{itemize}
Perform the same processing as for the other data, but now with $-5.0$ degrees phasing.
After integrating, the curve should look something like:
\begin{center}
\includegraphics[width=0.8\linewidth]{Figs/Fig3.png}
\end{center}
Now, there are multiple ways to continue, either you can export both data sets via File $\longrightarrow$ Export  $\longrightarrow$ ASCII, and perform further analysis using your favourite software, or we can continue within ssNake.
Here we will continue using ssNake for the analysis of the data.

We would like to obtain a curve of:
\begin{equation*}
  \Delta S = \frac{S_\text{wo} - S_\text{wi}}{S_\text{wo}}
\end{equation*}
for each point.
To do this, we duplicate of the no\_pulse data set:
\begin{itemize}
\item When on the `REDOR\_no\_pulse' data, use Workspaces $\longrightarrow$ Duplicate (and name it `Diff').
\end{itemize}
Now, we should subtract the data with the pulses from this data.
While in the `Diff' workspace:
\begin{itemize}
\item Combine $\longrightarrow$ Subtract (and select the `REDOR\_with\_pulse' data)
\end{itemize}
This should result in:
\begin{center}
\includegraphics[width=0.8\linewidth]{Figs/Fig4.png}
\end{center}
Now, we must divide the data by $S_\text{wo}$.
However, we should realize that, when dividing data, we do not want to perform a division of complex numbers (the integrals still contain the imaginary information).
Therefore, we should first take the real values of the data:
\begin{itemize}
\item In workspace `Diff' do Tools $\longrightarrow$ Real
\item In workspace `REDOR\_no\_pulse' do Tools $\longrightarrow$ Real
\item In workspace `Diff' do  Combine $\longrightarrow$ Divide, and select `REDOR\_no\_pulse'
\end{itemize}
This results in:
\begin{center}
\includegraphics[width=0.8\linewidth]{Figs/Fig5.png}
\end{center}
All values fall between 0 and 1 (as it should be).

Now, we can set the axis.
The first point was 2 rotor periods, the last 40. 
The spinning speed was 5 kHz, so a period is 0.2 ms.
This means that the axis should range from 0.4 to 8 ms:
\begin{itemize}
\item In workspace `Diff' use Plot $\longrightarrow$ User X-axis (Type: Linear, Start: 0.4e-3, Stop: 8e-3)
\end{itemize}
So the final figure should look like:
\begin{center}
\includegraphics[width=0.8\linewidth]{Figs/Fig6.png}
\end{center}
Which can be saved via File $\longrightarrow$ Export $\longrightarrow$ Figure.
Also note that this user axis is reset every time the axis needs to be re-calculated (in case of a Fourier transform for example).
This data is saved as `Final.mat' and is provide together with this tutorial.





\end{document}
