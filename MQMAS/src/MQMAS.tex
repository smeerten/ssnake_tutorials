% Copyright 2016 - 2018 Bas van Meerten and Wouter Franssen
%
%This file is part of ssNake.
%
%ssNake is free software: you can redistribute it and/or modify
%it under the terms of the GNU General Public License as published by
%the Free Software Foundation, either version 3 of the License, or
%(at your option) any later version.
%
%ssNake is distributed in the hope that it will be useful,
%but WITHOUT ANY WARRANTY; without even the implied warranty of
%MERCHANTABILITY or FITNESS FOR A PARTICULAR PURPOSE.  See the
%GNU General Public License for more details.
%
%You should have received a copy of the GNU General Public License
%along with ssNake. If not, see <http://www.gnu.org/licenses/>.

\documentclass[11pt,a4paper]{article}
\include{DeStijl}

\usepackage[bitstream-charter]{mathdesign}
\usepackage[T1]{fontenc}
\usepackage[protrusion=true,expansion,tracking=true]{microtype}
\pgfplotsset{compat=1.7,/pgf/number format/1000 sep={}, axis lines*=left,axis line style={gray},every outer x axis line/.append style={-stealth'},every outer y axis line/.append style={-stealth'},tick label style={font=\small},label style={font=\small},legend style={font=\footnotesize}}
\usepackage{colortbl}
\usetikzlibrary{calc}

%Set section font
\usepackage{sectsty}
\allsectionsfont{\color{black!70}\fontfamily{SourceSansPro-LF}\selectfont}
%--------------------


%Set toc fonts
\usepackage{tocloft}
%\renewcommand\cftchapfont{\fontfamily{SourceSansPro-LF}\bfseries}
\renewcommand\cfttoctitlefont{\color{black!70}\Huge\fontfamily{SourceSansPro-LF}\bfseries}
\renewcommand\cftsecfont{\fontfamily{SourceSansPro-LF}\selectfont}
%\renewcommand\cftchappagefont{\fontfamily{SourceSansPro-LF}\bfseries}
\renewcommand\cftsecpagefont{\fontfamily{SourceSansPro-LF}\selectfont}
\renewcommand\cftsubsecfont{\fontfamily{SourceSansPro-LF}\selectfont}
\renewcommand\cftsubsecpagefont{\fontfamily{SourceSansPro-LF}\selectfont}
%--------------------

%Define header/foot
%\usepackage{fancyhdr}
%\pagestyle{fancy}
%\fancyhead[LE,RO]{\fontfamily{SourceSansPro-LF}\selectfont \thepage}
%\fancyhead[LO,RE]{\fontfamily{SourceSansPro-LF}\selectfont \leftmark}
%\fancyfoot[C]{}
%--------------------

%remove page number from first chapter page
%\makeatletter
%\let\ps@plain\ps@empty
%\makeatother
%----------------------

\usepackage[hidelinks,colorlinks,allcolors=black, pdftitle={MQMAS processing in ssNake},pdfauthor={Wouter M.J.\ Franssen}]{hyperref}

\interfootnotelinepenalty=10000 %prevents splitting of footnote over multiple pages
\linespread{1.2}

\title{\color{black}\fontfamily{SourceSansPro-LF}\bfseries MQMAS processing in ssNake}
\author{}
\date{\color{black}\fontfamily{SourceSansPro-LF}\bfseries \today}


\begin{document}
%\newgeometry{left=72pt,right=72pt,top=95pt,bottom=95pt,footnotesep=0.5cm}
\microtypesetup{protrusion=true} % enables protrusion

\maketitle

\section{Introduction}
This tutorial will explain how Multiple Quantum Magic Angle spinning (MQMAS) NMR data can be processed using ssNake.
The tutorial delivered with the ssNake program is considered as prior knowledge.
If you have not yet studied this, please do so before continuing with this example.

MQMAS is a 2D experiment for half-integer quadrupolar nuclei which is used to obtain isotropic information from nuclei broadened by the second order quadrupole interaction.
This allows the separation of multiple overlapping quadrupolar sites on the basis of their `isotropic' value (which is a combination of the isotropic chemical shift, and the isotropic quadrupolar shift, which is also called the quadrupolar induced shift). 

The experiment can be quite difficult to process.
Not because of the complication of the steps involved, but due to the large number of different MQMAS experiments and the various ways to process these.
For this reason a full overview of all different approaches will not be given here.
However, an attempt is made to show a good method to process the data for both the regular type (shifting echo or $z$-filter) and the split t1 MQMAS experiment.


\section{Data}
In this tutorial we will use two data sets, recorded on solid rubidium nitrate (RbNO$_3$).
The sets are of a $^{87}$Rb $3Q$ MQMAS experiment.
One which was recorded with a $z$-filter and one using a split t1.
The data was recorded on a 600 MHz Varian machine using 15 kHz MAS.


\section{\textit{z}-filter processing}
First, we will look into the processing of MQMAS data recorded using a $z$-filter experiment (also called three pulse MQMAS).
Note that data recorded with a regular two pulse MQMAS (the standard MQMAS experiment) can be processed in the same way.

Processing the data is performed in the following steps:
\begin{itemize}
  \item Applying the hypercomplex operation (States)
  \item Processing of the direct dimensions D2 (zerofill, Fourier, apodize, phase)
  \item Processing of the indirect dimension D1 (zerofill, Fourier, apodize, phase)
  \item Shearing the spectrum
  \item Scaling the spectral width in the indirect dimension
  \item Referencing both dimensions
\end{itemize}

One of the tricky things of MQMAS processing is that the different traces along D1, have the echo maximum at different positions in the FID (along D2).
When applying apodization, the centre of the apodization window should be shifted differently for each trace along D1.


\begin{itemize}
	\item Open the Varian file Rb87\_3Q\_Zf.fid using File $\longrightarrow$ Open
	\item Set the view to D1 (sideframe, radiobutton)
	\item Convert the hypercomplex data via Transforms  $\longrightarrow$ Hypercomplex  $\longrightarrow$
	  States
	\item Set the view to D2 (sideframe, radiobutton)
\end{itemize}
Now, we will apply some apodization.
Note that in this case, the position of the echo maximum shift as a function of the D1 time.
This should be taken into account when applying the apodization, and is often called `shifting echo' apodization.

\begin{itemize}
	\item Open the Apodization window via Tools $\longrightarrow$ Apodize
 \item Set Gaussian at 80
 \item Set the Shifting type to `Spin 3/2, -3Q'. This sets the shifting parameter to 7/9 i.e. 0.7778
 \item Press Ok to apply.
\end{itemize}
The FID should now look like this:
\begin{center}
\includegraphics[width=0.8\linewidth]{Figs/Fig1.png}
\end{center}

\begin{itemize}
	\item Fourier Transform
	\item Phase with 2.5 degrees 0 order phasing (Tools  $\longrightarrow$ Phasing)
\end{itemize}

This results in:
\begin{center}
\includegraphics[width=0.8\linewidth]{Figs/Fig2.png}
\end{center}

Now, we should process the indirect dimension (D1):
\begin{itemize}
	\item Switch to D1 (sideframe, radiobutton), and select data point 614 in D2.
	\item Perform a complex conjugate (Tools $\longrightarrow$ Complex Conjugate)\footnote{The Varian data
	  we loaded has a different complex definition, so a conjugate in the indirect dimension is
	required (this depends on how the pulse sequence was programmed).}
\end{itemize}
This result in:
\begin{center}
\includegraphics[width=0.8\linewidth]{Figs/Fig3.png}
\end{center}
Now we should apply some zerofilling etc.:
\begin{itemize}
	\item Set the size to 1024 points (Matrix $\longrightarrow$ Sizing)
	\item Fourier Transform
\end{itemize}
This reesults in this spectrum along D1 for this trace:
\begin{center}
\includegraphics[width=0.8\linewidth]{Figs/Fig4.png}
\end{center}
The phase looks good, so we do not need to apply any phasing.

We have now processed both dimensions, so we can view the data as a contour plot:
\begin{itemize}
	\item Switch to D2 (sideframe, radiobutton)
	\item Change the view to a contour plot: Plot  $\longrightarrow$ Contour
\end{itemize}
We now have:
\begin{center}
\includegraphics[width=0.8\linewidth]{Figs/Fig5.png}
\end{center}
And zoomed:
\begin{center}
\includegraphics[width=0.8\linewidth]{Figs/Fig6.png}
\end{center}
Note that the right projection was changed to `Max' instead of the standard `Sum'.
This results in a skyline projection, which is affected less by the noise regions of the data.

As can be seen, the different powder patterns are all tilted.
This is expected for a $z$-filter MQMAS, and can be corrected by shearing the spectrum.
\begin{itemize}
	\item Shear via Matrix $\longrightarrow$ Shearing, using the `Spin 3/2, -3Q' setting, and
	  direction `1' and axis '2'
\end{itemize}
This results in (zoomed):
\begin{center}
\includegraphics[width=0.8\linewidth]{Figs/Fig7.png}
\end{center}
Which is a nice spectrum! This is the final figure, we now only need to fix the axes.

By processing the spectrum in this way, we have removed the second order quadrupolar broadening from the indirect dimension.
This dimension is therefore referred to as the isotropic dimension.
The position along this axis is determined by the scaled isotropic chemical shift, and the scaled quadrupole induced shift.
However, it is more convenient to have a frequency axis which shows the unscaled isotropic chemical shift.
To accomplishing this, we must scale the spectral width by a specific value (which depends on the spin quantum number and the MQ transition of the experiment).
\begin{itemize}
	\item Switch to D1 (sideframe, click on the upper left radiobutton)
	\item Use Tools  $\longrightarrow$ Scale SW, and select `Spin 3/2, -3Q' and apply
\end{itemize}
Now, we have scaled the axis.
As a last step, we should apply the chemical shift reference.
This was determined using a rubidium nitrate solution.
Based on this the 0 ppm frequency is at: 196.3182865 MHz.
\begin{itemize}
  \item Set the reference via Tools $\longrightarrow$ Reference $\longrightarrow$ Set Reference, and
	 put `196.3182865' in the Frequency box
	\item Switch to D2 (sideframe, click on the lower left radiobutton)
	\item Set the reference in the same way
\end{itemize}
The final spectrum should now look like this (zoomed):
\begin{center}
\includegraphics[width=0.8\linewidth]{Figs/Fig8.png}
\end{center}
This spectrum is also delivered with this tutorial, and named `Rb87\_3Q\_Zf\_(final\_spectrum).mat'.
Note that I extracted the relevant region before saving (to reduce size of the file).
 
Now, we can fit this spectrum.
Either by fitting a specific trace with a regular second order quadrupolar line, or by fitting the entire MQMAS spectrum.
Another alternative is to only determine the isotropic shift and the quadupolar product $P_\text{Q} = C_\text{Q}\sqrt{1+\eta^2/3}$.
For this, we must determine the Centre of Mass for each site, in both D1 and D2.
This can be done by going to the relevant trace, and using Fitting  $\longrightarrow$ Centre of Mass.
Note that in D1, the peaks are symmetric, so the highest point is the centre in this case.

\begin{center}
  \begin{tabular}{c|cc| c c}
	 \toprule
	 Site & $\delta_1$ [ppm]& $\delta_2$ [ppm]& $\delta_\text{iso}$ [ppm]& $P_\text{Q}$ [MHz]  \\
	 \midrule
	 1 & $-30.5$ & $-34.17$ & $-31.86$ & 1.89 \\
	 2& $-26.8$ &$-31.96$  &$-28.71$ & 2.24\\
	 3& $-26.3$ &$-29.76$ & $-27.58$ & 1.83 \\
	\bottomrule
  \end{tabular}
\end{center}
The last two columns have been calculated using the MQMAS Extraction utility (see the Utilities menu). 


\section{Split \textit{T}$_\text{1}$ processing}
In a split $T_\text{1}$ experiment, the additional shift of the echo is taken into account within the pulse sequence.
This leads to regular echo data, were the position of the echo in the time domain is always the same.
Due to this, the shearing is no longer required.
Also, the data is recorded in such a way that no hypercomplex processing is necessary.
The following assumes that you have read the information above (about the $z$-filter processing).

\begin{itemize}
	\item Open the Varian file Rb87\_3Q\_splitt1.fid using File $\longrightarrow$ Open
\end{itemize}
Now, we must process this data as a whole echo acquisition (see the tutorial on this).

\begin{itemize}
	\item Swap the echo at position 375 (Tools $\longrightarrow$ Swap Echo)
	\item Set the size to 4096 points (Matrix $\longrightarrow$ Sizing)
	\item Apply apodization if required (not used in this case)
	\item Fourier Transform
	\item Phase the imaginary part to zero (168.9 degrees, via Tools $\longrightarrow$ Phasing)
\end{itemize}
This should result in (zoomed):
\begin{center}
\includegraphics[width=0.8\linewidth]{Figs/Fig9.png}
\end{center}
Now we can process D1:

\begin{itemize}
	\item Switch to D1 (sideframe, radiobutton)
	\item View position 2081 along D2
	\item Tools $\longrightarrow$  Complex Conjugate
	\item Set size to 512
	\item Fourier Transform
\end{itemize}
This results in:
\begin{center}
\includegraphics[width=0.8\linewidth]{Figs/Fig10.png}
\end{center}

Now, me must scale the spectral width of this dimension, as described above.
However, with this experiment something strange is going on.
According to the Varian pulse sequence used to record this data, the SW that you supply is $9/16$ times the desired spectral width.
For the processing, this means that this scaling should first be undone.

\begin{itemize}
	\item Multiply the spectral width by 16.0/9.0 (either via Tools  $\longrightarrow$ Scale SW, or
	  by typing at the SW box in the bottom frame).
	\item Multiply the SW by the MQMAS scaling value: Tools  $\longrightarrow$ Scale SW, and select `Spin 3/2, -3Q'
\end{itemize}
Now, we should reference the ppm axis in both dimension.
As before:
\begin{itemize}
  \item Set the reference via Tools $\longrightarrow$ Reference $\longrightarrow$ Set Reference, and
	 fill in `196.3182865' in the Frequency box
	\item Switch to D2 (sideframe, click on the lower left radiobutton)
	\item Set the reference in the same way
\end{itemize}
Switching to a contour plot results in (zoomed):
\begin{center}
\includegraphics[width=0.8\linewidth]{Figs/Fig11.png}
\end{center}
This is equivalent to the spectrum obtained before, for the $z$-filter data.
This spectrum is also supplied together with this tutorial, and named `Rb87\_3Q\_splitt1\_(final\_spectrum).mat'.
Note that the relevant region was extracted (to reduce the size of the file).


\section{Equations}
The following Section will show some equations for the relevant shearing and scaling constants used
for MQMAS processing.\footnote{This sections is based on:
  P.\ P.\ Man, \textit{Phys.\ Rev.\ B}, \textbf{5}, 2764 (1998) and 
  T.\ Anup\~{o}ld, A.\ Reinhold, P.\ Sarv, A.\ Samoson, \textit{Solid State Nucl.\ Magn.\ Reson.},
  \textbf{13}, 87 (1998).}

These are all included in ssNake in such a way that there is no need to remember these values.
However, for the sake of completeness, they are provided here.

The following table summarises the values:
\begin{center}
\begin{tabular}[h]{c r r r r}
  \toprule
  $I$ & p$Q$ & $k$ & 1/$a$ & $z$\\
  \midrule
  3/2 & $-3Q$ & 7/9 & 9/34 & 680/27\\
  5/2 & $3Q$ & 19/12 & $-12/17$& 8500/81\\
   & $-5Q$ & 25/12 &12/85& 8500/81 \\
  7/2 & $3Q$ & 101/45 & $-45/34$& 6664/27 \\
   & $5Q$ & 11/9 & $-9/34$ & 6664/27\\
   & $-7Q$ & 161/45 & $45/476$ & 6664/27\\
  9/2 & $3Q$ & 91/36 &  $-36/17$ & 1360/3 \\
   & $5Q$ & 95/36 & $-36/85$ & 1360/3 \\
	& $7Q$ & 7/18 & $-18/117$ & 1360/3 \\
	& $-9Q$ & 31/6 & 6/85 & 1360/3 \\
  \bottomrule
\end{tabular}
\end{center}
Here, $k$ is the shearing factor, $1/a$ the scaling of the spectral width, and $z$ is a value required to determine the quadrupolar product $P_\text{Q}$ from an MQMAS spectrum (see later).
The equations are:
\begin{equation}
  k = p \frac{36I(I+1)-17p^2 - 10}{36I(I+1) - 27}
\end{equation}
\begin{equation}
  1/a = 1/(k - p)
\end{equation}
  \begin{equation}
	 z = \frac{1}{\frac{b}{a}-r}
  \end{equation}
  With:
 \begin{equation}
	b = r	(k + \lambda)
 \end{equation}
 and
 \begin{equation}
	r = -\frac{3}{10}\frac{I(I+1)-3/4}{[2I(2I-1)]^2}
 \end{equation}
 \begin{equation}
	\lambda = p \frac{I(I+1)-3/4 \cdot p^2}{-I(I+1)+3/4}
 \end{equation}

 \subsection{Further background}
 Here, we will quickly derive the relevant equations shown above.

 The centre of mass of a line in the the MQMAS spectrum is located in D2 at:
 \begin{equation}
	\delta_\text{D2} = \delta_\text{iso} + \delta_\text{QIS}  =  \delta_\text{iso} + r
	\frac{P_\text{Q}^2}{\nu_0^2} \cdot 10^6
 \end{equation}
 with
 \begin{equation}
	r = -\frac{3}{10}\frac{I(I+1)-3/4}{[2I(2I-1)]^2}
 \end{equation}
 In D1 the centre of mass, before shearing is located at:
 \begin{equation}
	\delta_\text{D1} = -p\delta_\text{iso} + \delta_\text{QIS}  =  -p\delta_\text{iso} +
	\lambda r
	\frac{P_\text{Q}^2}{\nu_0^2} \cdot 10^6
 \end{equation}
 with:

 \begin{equation}
	\lambda = p \frac{I(I+1)-3/4 \cdot p^2}{-I(I+1)+3/4}
 \end{equation}
After shearing, the centre of mass gets shifted to:
 \begin{align}
	\delta_\text{D1'} &= \delta_\text{D1} + k\delta_\text{D2} = (k - p)\delta_\text{iso} +
	r	(k + \lambda) \frac{P_\text{Q}^2}{\nu_0^2} \cdot 10^6 \\
	& = a  \delta_\text{iso} + b \frac{P_\text{Q}^2}{\nu_0^2} \cdot 10^6
 \end{align}
 with constants:
 \begin{align}
	a & = k(I,p) - p \\
	b & = r(I)	(k(I,p) + \lambda(I,p))
 \end{align}
 A good way to processes this data, is to scale the spectral width of F1 by $1/a$.
 This means in the resulting spectrum, changes in chemical shift will be along the diagonal of the spectrum.
This leads to:
\begin{equation}
  \delta_\text{D1''} = \delta_\text{iso} + \frac{b}{a} \frac{P_\text{Q}^2}{\nu_0^2} \cdot 10^6
\end{equation}
When this processing is performed, nuclei which have a quadrupolar coupling are always located in the lower right part of the spectrum, beneath the diagonal that is.
When a lineshape is located on the diagonal, and stretched along it, this is caused by a distribution in chemical shift.

Based on the centre of mass in D1'' and D2, the NMR parameters $\delta_\text{iso}$ and $P_\text{Q}$ can be determined:

\begin{align}
  \delta_\text{iso} & = \delta_\text{D1''} - \frac{b}{a\cdot r} \delta_\text{D2} =
                      \delta_\text{D1''} - \frac{k+\lambda}{a} \delta_\text{D2} \\
                    & = \frac{17 \delta_\text{D1''} + 10  \delta_\text{D2}}{27}
\end{align}
This value is independent of $I$ and $p$.

For $P_\text{Q}$ the calculation is a bit more difficult:
\begin{align}
  \delta_\text{D1''} - \delta_\text{D2} =& \left( \frac{b}{a} - r \right)
                                           \frac{P_\text{Q}^2}{\nu_0^2} \cdot 10^6\\ 
  P_\text{Q} =& \sqrt{  \frac{1}{\frac{b}{a}-r}\cdot 10^{-6} \nu_0^2 (\delta_\text{F1''} -
                \delta_\text{F2}) } \\
  =& \sqrt{  z \cdot 10^{-6} \nu_0^2 (\delta_\text{F1''} -
     \delta_\text{F2}) } 
\end{align}
With a scaling factor of:
\begin{equation}
  z = \frac{1}{\frac{b}{a}-r}
\end{equation}
These depend only on $I$ and are tabulated above.
These equation can be used to calculate the isotropic shift and quadrupolar product ($P_\text{Q}$) of a line based on the centre of mass of the line in both dimension.
These equations have been included in ssNake in the form of a utility, which can by found in the Utilities menu.

\end{document}
